\section{Let's do some math}

\begin{figure}[H]
    \centering
    \includegraphics[scale=0.65]{pi}
\end{figure}

\subsection{Number of possible keys}
    
Let's note $\Psi(n,C,u)$ the number of possible of keys. Then :
\begin{large}
\begin{align}
    \Psi(n,C,u) = K_n \times C! \times (2^{u-2}-1)
\end{align}
\end{large}
With :
\begin{itemize}
    \item $K_{n}$ the number of possible keys with n "bytes", e.g. $K_{100}$ = fff...f (200 times) - 100...0 (1, and 0 199 times)
    \item $C$ the number of chars in the charset (not including the null chars, so this part may be slightly bigger)
    \item $u$ is the number of unused chars. This part comes from 
                \href{https://math.stackexchange.com/questions/3340723/how-many-ways-to-partition-n-elements-into-two-nonempty-subsets}{here} 
          and we do not count the split char
    
\end{itemize}

\subsection{Number of possible encrypted message with the same message and key}

Let's note $\mathcal{M}(C)$ the number of possible of keys. Then :
\begin{large}
\begin{align}
    \mathcal{M}(C) = 2 \sum_{k=0}^{m}{C^k} 
\end{align}
\end{large}
With, again, $C$ the number of chars in the charset, and $m$ the upper bound.