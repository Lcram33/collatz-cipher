\section{Introdution}

\begin{figure}[H]
    \centering
    \includegraphics[scale=0.175]{bulb}
\end{figure}

First, a bit of context and reminder.

\subsection{Symmetric cipher}

Symmetric ciphers use symmetric algorithms to encrypt and decrypt data. They use the same key to encrypt and decrypt data.
They differ from asymmetric ciphers, which use two keys : one to encrypt, and the other one to decrypt. The former reduces 
the data to send via a secure channel, while the latter allow to establish a secure channel without having any.

\subsection{Collatz conjecture}

AKA :

\begin{itemize}
    \item $3n + 1$ conjecture
    \item Ulam conjecture
    \item Hasse's algorithm
    \item Syracuse problem
    \item ... and a bunch of other names.
\end{itemize}
One of the most famous unsolved problems in mathematics. Consider the following operation on an arbitrary positive 
integer :

\begin{itemize}
    \item If the number is even, divide it by two.
    \item If the number is odd, triple it and add one.
\end{itemize}
Doing this repeateadly always gives 1, and then goes in an infinite loop $(1 \rightarrow 4 \rightarrow 2 \rightarrow 1 
\rightarrow ...)$, no matter the number you start with (for any positive integer, zero excluded).
\newline
\newline
Let's define this mathematically.
\newline
\newline
We consider the function $f$ such as :

\[
    f(n)=\left\{
    \begin{array}{ll}
        \dfrac{n}{2} $ if $ n \equiv 0 $ mod $ 2\\
        \\
        3n+1 $ if $ n \equiv 1 $ mod $ 2
    \end{array}
    \right.
\]
And we can form the following sequence, with any integer $n \geq 1$ :

\begin{large}
    \[
        a_i=\left\{
            \begin{array}{ll}
                n & $ if $ i = 0\\
                f(a_ {i-1}) & $ if $ i > 0
            \end{array}
            \right.
            \]
\end{large}
The Collatz conjecture is : this process will eventually reach the number 1, regardless of which positive integer is chosen initially.

\begin{large}    
\begin{align}
    \forall n \geq 1 / a_0 = n, \exists \varepsilon \geq 0 / a_{\varepsilon} = 1
\end{align} 
\end{large}